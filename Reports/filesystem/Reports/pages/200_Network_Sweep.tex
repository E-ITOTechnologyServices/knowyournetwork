%    Copyright (C) 2017 e-ito Technology Services GmbH
%    e-mail: info@e-ito.de
%
%    This program is free software: you can redistribute it and/or modify
%    it under the terms of the GNU General Public License as published by
%    the Free Software Foundation, either version 3 of the License, or
%    (at your option) any later version.
%
%    This program is distributed in the hope that it will be useful,
%    but WITHOUT ANY WARRANTY; without even the implied warranty of
%    MERCHANTABILITY or FITNESS FOR A PARTICULAR PURPOSE.  See the
%    GNU General Public License for more details.
%
%    You should have received a copy of the GNU General Public License
%    along with this program.  If not, see <http://www.gnu.org/licenses/>.

\chapter[Network Sweep (Restricted Ports)]{\underline{Network Sweep (Restricted Ports)}}
\begin{flushleft}
The following table list systems that start network sweeps by initiating connections to more than 200 destinations in a
short period of time using one out of a set of restricted ports (25 – smtp, 53 – dns, 80 – http, 443 – https, 445 – netbios
and 3389 – RDP). These systems might be compromised hosts starting a network reconnaissance.
\end{flushleft}

\begin{table}[H]
  \begin{center}
    \pgfplotstabletypeset[
      font=\small,
      col sep=comma, % the seperator in our .csv file
      use comma, % Decimal separator
      every even row/.style={ before row={\rowcolor {green4}}},
      every head row/.style={ before row=\toprule,after row=\midrule},
      every last row/.style={ after row=\bottomrule},
      columns/Source IP/.style={column type=l,string type,column name=Source IP Address},
      columns/Number of Connections/.style={int detect,column type=r,string type,column name=Number of Connections},
      multicolumn names
    ]{csv/200_Network_Sweep.csv} % filename/path to file
    \caption[\normalsize{Indicator of Network Sweeps}]{\small{Indicator of Network Sweeps}}
  \end{center}
\end{table}
